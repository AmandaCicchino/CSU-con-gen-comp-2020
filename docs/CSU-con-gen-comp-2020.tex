\documentclass[]{book}
\usepackage{lmodern}
\usepackage{amssymb,amsmath}
\usepackage{ifxetex,ifluatex}
\usepackage{fixltx2e} % provides \textsubscript
\ifnum 0\ifxetex 1\fi\ifluatex 1\fi=0 % if pdftex
  \usepackage[T1]{fontenc}
  \usepackage[utf8]{inputenc}
\else % if luatex or xelatex
  \ifxetex
    \usepackage{mathspec}
  \else
    \usepackage{fontspec}
  \fi
  \defaultfontfeatures{Ligatures=TeX,Scale=MatchLowercase}
\fi
% use upquote if available, for straight quotes in verbatim environments
\IfFileExists{upquote.sty}{\usepackage{upquote}}{}
% use microtype if available
\IfFileExists{microtype.sty}{%
\usepackage{microtype}
\UseMicrotypeSet[protrusion]{basicmath} % disable protrusion for tt fonts
}{}
\usepackage{hyperref}
\hypersetup{unicode=true,
            pdftitle={Advanced Computing and Bioinformatics for Conservation Genomics. CSU FWCB/BZ 5XX},
            pdfauthor={Instructors: Eric C. Anderson and Kristen C. Ruegg},
            pdfborder={0 0 0},
            breaklinks=true}
\urlstyle{same}  % don't use monospace font for urls
\usepackage{natbib}
\bibliographystyle{apalike}
\usepackage{longtable,booktabs}
\usepackage{graphicx,grffile}
\makeatletter
\def\maxwidth{\ifdim\Gin@nat@width>\linewidth\linewidth\else\Gin@nat@width\fi}
\def\maxheight{\ifdim\Gin@nat@height>\textheight\textheight\else\Gin@nat@height\fi}
\makeatother
% Scale images if necessary, so that they will not overflow the page
% margins by default, and it is still possible to overwrite the defaults
% using explicit options in \includegraphics[width, height, ...]{}
\setkeys{Gin}{width=\maxwidth,height=\maxheight,keepaspectratio}
\IfFileExists{parskip.sty}{%
\usepackage{parskip}
}{% else
\setlength{\parindent}{0pt}
\setlength{\parskip}{6pt plus 2pt minus 1pt}
}
\setlength{\emergencystretch}{3em}  % prevent overfull lines
\providecommand{\tightlist}{%
  \setlength{\itemsep}{0pt}\setlength{\parskip}{0pt}}
\setcounter{secnumdepth}{5}
% Redefines (sub)paragraphs to behave more like sections
\ifx\paragraph\undefined\else
\let\oldparagraph\paragraph
\renewcommand{\paragraph}[1]{\oldparagraph{#1}\mbox{}}
\fi
\ifx\subparagraph\undefined\else
\let\oldsubparagraph\subparagraph
\renewcommand{\subparagraph}[1]{\oldsubparagraph{#1}\mbox{}}
\fi

%%% Use protect on footnotes to avoid problems with footnotes in titles
\let\rmarkdownfootnote\footnote%
\def\footnote{\protect\rmarkdownfootnote}

%%% Change title format to be more compact
\usepackage{titling}

% Create subtitle command for use in maketitle
\providecommand{\subtitle}[1]{
  \posttitle{
    \begin{center}\large#1\end{center}
    }
}

\setlength{\droptitle}{-2em}

  \title{Advanced Computing and Bioinformatics for Conservation Genomics. CSU FWCB/BZ 5XX}
    \pretitle{\vspace{\droptitle}\centering\huge}
  \posttitle{\par}
    \author{\textbf{Instructors:} Eric C. Anderson and Kristen C. Ruegg}
    \preauthor{\centering\large\emph}
  \postauthor{\par}
      \predate{\centering\large\emph}
  \postdate{\par}
    \date{\textbf{Site Last Updated:} 2019-12-05}

\usepackage{booktabs}
\usepackage{amsthm}
\makeatletter
\def\thm@space@setup{%
  \thm@preskip=8pt plus 2pt minus 4pt
  \thm@postskip=\thm@preskip
}
\makeatother

\begin{document}
\maketitle

{
\setcounter{tocdepth}{1}
\tableofcontents
}
\hypertarget{course-overview}{%
\chapter*{Course Overview}\label{course-overview}}
\addcontentsline{toc}{chapter}{Course Overview}

Welcome to the course web page for Advanced Computing and Bioinformatics for Conservation Genomics,
an experimental course being offered for the first time in Winter Semester 2020. We will be covering
computing, analysis and data-organization strategies for bioinformatics and analysis of high-throughput sequencing data for ecology,
evolution, and conservation.

Modern high-throughput sequencing can provide extraordinary amounts of data, enabling researchers to tackle a wide range of questions and problems in ecology, evolution, conservation, and fisheries and wildlife management. Preparing and processing these data for use, however, requires multiple bioinformatic steps, and subsequent analysis of these large, complex data sets must rely on specialized computer programs. Mastering these skills presents a high bar for students originating from outside of computer science and related fields. At present, in many institutions, such skills are typically learned from peers within experienced laboratories, or through a variety of workshops. This course aims to comprehensively teach the computing and analytical skills necessary to use genomic data from high-throughput sequencing in the context of ecological research. During the first 2/3 of the course, the focus is on aligning DNA sequence data and identifying variants across multiple individuals. In the last 1/3 of the course we consider a series of case studies in how such data are used to make inference for applications in fisheries, wildlife, and conservation. Outside of the bioinformatic utilities that run within a Unix framework, emphasis is placed on using the R programming language and RStudio for project management and documentation.

The proposed course
topics appear, by week, in the table below. Each week of the course is structured as a different
chapter in the navigation panel on the left. In order to figure out what we are doing in the course
each week, that will be the first place to check. The week's objectives, readings, and exercises will
be listed there.

\hypertarget{course-learning-objectives}{%
\section*{Course Learning Objectives}\label{course-learning-objectives}}
\addcontentsline{toc}{section}{Course Learning Objectives}

Upon successful completion of the course, students will be able to:

\begin{enumerate}
\def\labelenumi{\arabic{enumi}.}
\tightlist
\item
  Organize and execute a complex bioinformatic data-analysis project in a manner that makes it easily understood and reproduced by others.
\item
  Describe the main data formats used in genomic analysis, and know how to generate and manipulate them.
\item
  Work with a wide range of the bioinformatic tools available in the Unix environment and understand how to script these tools into pipelines for DNA sequence alignment, variant calling, and analysis.
\item
  Understand how to break down complex genomic analysis projects into small, independent chunks and execute those using job arrays on a high performance computing cluster.
\item
  Perform a variety of computational analyses central to conservation genetics.
\end{enumerate}

\hypertarget{assessment}{%
\section*{Assessment}\label{assessment}}
\addcontentsline{toc}{section}{Assessment}

Assessment will be based mostly on weekly problem sets. These will sometimes require
considerable time and thought, but they will be critical for solidifying the concepts
and procedures in the course. Students will also be undertaking individual projects
in which they apply the skills they have learned in the course to a data set
relevant in some way to their own research or to an interesting question relevant
to some existing data, after discussion with the instructors (see \protect\hyperlink{indproj}{below}).
Finally, students are expected to contribute to discussion and participation in the course,
including (and most importantly) being helpful to one another in order to learn challenging material,
together, in a supportive environment.

\begin{quote}
``It is literally true that you can succeed best
and quickest by helping others to succeed.''

\hfill ---- Napolean Hill
\end{quote}

~

\begin{longtable}[]{@{}ll@{}}
\toprule
Assessment Component & Percentage of Grade\tabularnewline
\midrule
\endhead
Problem sets & 60\%\tabularnewline
Individual analysis projects & 30\%\tabularnewline
Class participation & 10\%\tabularnewline
\bottomrule
\end{longtable}

\hypertarget{weekly-schedule}{%
\section*{Weekly Schedule}\label{weekly-schedule}}
\addcontentsline{toc}{section}{Weekly Schedule}

The schedule is subject to change as the semester proceeds, but this
is what we are shooting for.

\begin{longtable}[]{@{}lll@{}}
\toprule
\begin{minipage}[b]{0.13\columnwidth}\raggedright
Week\strut
\end{minipage} & \begin{minipage}[b]{0.46\columnwidth}\raggedright
Lecture Component\strut
\end{minipage} & \begin{minipage}[b]{0.33\columnwidth}\raggedright
Lab Component\strut
\end{minipage}\tabularnewline
\midrule
\endhead
\begin{minipage}[t]{0.13\columnwidth}\raggedright
1\strut
\end{minipage} & \begin{minipage}[t]{0.46\columnwidth}\raggedright
Rstudio Projects, GitHub\strut
\end{minipage} & \begin{minipage}[t]{0.33\columnwidth}\raggedright
Rmarkdown, git and GitHub\strut
\end{minipage}\tabularnewline
\begin{minipage}[t]{0.13\columnwidth}\raggedright
2\strut
\end{minipage} & \begin{minipage}[t]{0.46\columnwidth}\raggedright
Unix, directory structure, utilities\strut
\end{minipage} & \begin{minipage}[t]{0.33\columnwidth}\raggedright
Unix, data compression\strut
\end{minipage}\tabularnewline
\begin{minipage}[t]{0.13\columnwidth}\raggedright
3\strut
\end{minipage} & \begin{minipage}[t]{0.46\columnwidth}\raggedright
Next generation sequencing, alignment conventions, FASTA, SAM\strut
\end{minipage} & \begin{minipage}[t]{0.33\columnwidth}\raggedright
Samtools (faidx)\strut
\end{minipage}\tabularnewline
\begin{minipage}[t]{0.13\columnwidth}\raggedright
4\strut
\end{minipage} & \begin{minipage}[t]{0.46\columnwidth}\raggedright
Shell scripting and awk\strut
\end{minipage} & \begin{minipage}[t]{0.33\columnwidth}\raggedright
Shell scripting and awk\strut
\end{minipage}\tabularnewline
\begin{minipage}[t]{0.13\columnwidth}\raggedright
5\strut
\end{minipage} & \begin{minipage}[t]{0.46\columnwidth}\raggedright
Remote computers, ssh, rclone, HPC, SLURM, SGE\strut
\end{minipage} & \begin{minipage}[t]{0.33\columnwidth}\raggedright
Connecting to Summit Cluster\strut
\end{minipage}\tabularnewline
\begin{minipage}[t]{0.13\columnwidth}\raggedright
6\strut
\end{minipage} & \begin{minipage}[t]{0.46\columnwidth}\raggedright
Sequence alignment, job arrays, parallelization over individuals\strut
\end{minipage} & \begin{minipage}[t]{0.33\columnwidth}\raggedright
bwa, samtools (sort, merge, cat, index)\strut
\end{minipage}\tabularnewline
\begin{minipage}[t]{0.13\columnwidth}\raggedright
7\strut
\end{minipage} & \begin{minipage}[t]{0.46\columnwidth}\raggedright
Variant calling -- I, fundamental concepts\strut
\end{minipage} & \begin{minipage}[t]{0.33\columnwidth}\raggedright
GATK, Base quality score recalibration\strut
\end{minipage}\tabularnewline
\begin{minipage}[t]{0.13\columnwidth}\raggedright
8\strut
\end{minipage} & \begin{minipage}[t]{0.46\columnwidth}\raggedright
Variant calling -- II, parallelization over regions\strut
\end{minipage} & \begin{minipage}[t]{0.33\columnwidth}\raggedright
GATK, gVCFs, VCFs\strut
\end{minipage}\tabularnewline
\begin{minipage}[t]{0.13\columnwidth}\raggedright
9\strut
\end{minipage} & \begin{minipage}[t]{0.46\columnwidth}\raggedright
Variant Filtering, manipulation, exploration, LD and pop-gen statistics\strut
\end{minipage} & \begin{minipage}[t]{0.33\columnwidth}\raggedright
bcftools, bedtools, R package `whoa', plink\strut
\end{minipage}\tabularnewline
\begin{minipage}[t]{0.13\columnwidth}\raggedright
10\strut
\end{minipage} & \begin{minipage}[t]{0.46\columnwidth}\raggedright
Restriction-associated digest (RAD) sequencing\strut
\end{minipage} & \begin{minipage}[t]{0.33\columnwidth}\raggedright
STACKS2, R package `radiator'\strut
\end{minipage}\tabularnewline
\begin{minipage}[t]{0.13\columnwidth}\raggedright
11\strut
\end{minipage} & \begin{minipage}[t]{0.46\columnwidth}\raggedright
Visualization of genomic data in space\strut
\end{minipage} & \begin{minipage}[t]{0.33\columnwidth}\raggedright
R packages: `ggplot', `sf'\strut
\end{minipage}\tabularnewline
\begin{minipage}[t]{0.13\columnwidth}\raggedright
12\strut
\end{minipage} & \begin{minipage}[t]{0.46\columnwidth}\raggedright
Visualization of genomic data: trees\strut
\end{minipage} & \begin{minipage}[t]{0.33\columnwidth}\raggedright
R packages: `ape', `ggtree'\strut
\end{minipage}\tabularnewline
\begin{minipage}[t]{0.13\columnwidth}\raggedright
13\strut
\end{minipage} & \begin{minipage}[t]{0.46\columnwidth}\raggedright
Population structure\strut
\end{minipage} & \begin{minipage}[t]{0.33\columnwidth}\raggedright
R package `srsStuff'\strut
\end{minipage}\tabularnewline
\begin{minipage}[t]{0.13\columnwidth}\raggedright
14\strut
\end{minipage} & \begin{minipage}[t]{0.46\columnwidth}\raggedright
Inbreeding, runs of homozygosity\strut
\end{minipage} & \begin{minipage}[t]{0.33\columnwidth}\raggedright
bcftools roh, plink\strut
\end{minipage}\tabularnewline
\begin{minipage}[t]{0.13\columnwidth}\raggedright
15\strut
\end{minipage} & \begin{minipage}[t]{0.46\columnwidth}\raggedright
Genome wide association studies\strut
\end{minipage} & \begin{minipage}[t]{0.33\columnwidth}\raggedright
ANGSD, snpEff\strut
\end{minipage}\tabularnewline
\begin{minipage}[t]{0.13\columnwidth}\raggedright
16\strut
\end{minipage} & \begin{minipage}[t]{0.46\columnwidth}\raggedright
Genotype-environment association\strut
\end{minipage} & \begin{minipage}[t]{0.33\columnwidth}\raggedright
Gradient Random Forest, RDA\strut
\end{minipage}\tabularnewline
\bottomrule
\end{longtable}

\hypertarget{indproj}{%
\section*{Individual Projects}\label{indproj}}
\addcontentsline{toc}{section}{Individual Projects}

The purpose of the individual projects is to allow the students to use
many of the skills learned, and to gain experience in preparing a reproducible
research project. Some students likely already have their own data sets
that they are working on, but we expect that many will not. We will be
able to provide data and interesting questions to tackle from our own research.
Additionally, we will encourage students to take on related projects so that
they can work together on different parts of a single question.

\hypertarget{getting-setup-with-rstudio-and-github}{%
\chapter{Getting setup with Rstudio and GitHub}\label{getting-setup-with-rstudio-and-github}}

It might seem strange to start a bioinformatics course with a segment
on RStudio, since most bioinformatics takes place on large computer
clusters which can't always be configured by individual users to
play well with RStudio running on a server. Not to mention the fact that
R is not typically central to hard-core bioinformatic tasks in a
Unix environment. However, bear with us. Using RStudio sets you up
with a great mechanism for recording your work in RMarkdown Notebooks,
it provides a natural way to organize projects, it makes it trivial
to keep your work version-controlled, and finally, when you have done
all your bioinformatics, it is great for analysis of and figure preparation from
the generated data in the R language.

\begin{center}\rule{0.5\linewidth}{\linethickness}\end{center}

\hypertarget{required-readings}{%
\section*{Required Readings}\label{required-readings}}
\addcontentsline{toc}{section}{Required Readings}

\begin{enumerate}
\def\labelenumi{\arabic{enumi}.}
\tightlist
\item
  This
\item
  That
\item
  Etc
\end{enumerate}

\hypertarget{recommended-readings}{%
\section*{Recommended Readings}\label{recommended-readings}}
\addcontentsline{toc}{section}{Recommended Readings}

\begin{enumerate}
\def\labelenumi{\arabic{enumi}.}
\tightlist
\item
  This
\item
  That
\item
  More
\end{enumerate}

\hypertarget{exercises-due-by-mmddyyy}{%
\section*{Exercises due by MM/DD/YYY}\label{exercises-due-by-mmddyyy}}
\addcontentsline{toc}{section}{Exercises due by MM/DD/YYY}

\begin{enumerate}
\def\labelenumi{\arabic{enumi}.}
\tightlist
\item
  Links to notebooks that have the assignments. Will make it
  easy for students to download extra material for each one.
\end{enumerate}

\begin{center}\rule{0.5\linewidth}{\linethickness}\end{center}

\hypertarget{first-topic-for-lecture}{%
\section{First topic for lecture}\label{first-topic-for-lecture}}

\hypertarget{second-topic-for-lecture}{%
\section{Second topic for lecture}\label{second-topic-for-lecture}}

Boing

\hypertarget{a-one-week-unix-crash-course}{%
\chapter{A one-week Unix crash course}\label{a-one-week-unix-crash-course}}

Preamble, preamble, preamble.

\begin{center}\rule{0.5\linewidth}{\linethickness}\end{center}

\hypertarget{required-readings-1}{%
\section*{Required Readings}\label{required-readings-1}}
\addcontentsline{toc}{section}{Required Readings}

\begin{enumerate}
\def\labelenumi{\arabic{enumi}.}
\tightlist
\item
  \href{https://eriqande.github.io/eca-bioinf-handbook/essential-unixlinux-terminal-knowledge.html}{ECA-bioinf-handbook-Chapter-4}. Read the whole
  thing before coming to class on Tuesday, and commit the terms in the ``Unix study guide'' (the long table at the end) to memory.
\end{enumerate}

\hypertarget{recommended-readings-1}{%
\section*{Recommended Readings}\label{recommended-readings-1}}
\addcontentsline{toc}{section}{Recommended Readings}

\begin{enumerate}
\def\labelenumi{\arabic{enumi}.}
\tightlist
\item
  This
\item
  That
\item
  More
\end{enumerate}

\hypertarget{exercises-due-by-mmddyyy-1}{%
\section*{Exercises due by MM/DD/YYY}\label{exercises-due-by-mmddyyy-1}}
\addcontentsline{toc}{section}{Exercises due by MM/DD/YYY}

\begin{enumerate}
\def\labelenumi{\arabic{enumi}.}
\tightlist
\item
  Links to notebooks that have the assignments. Will make it
  easy for students to download extra material for each one.
\end{enumerate}

\begin{center}\rule{0.5\linewidth}{\linethickness}\end{center}

\hypertarget{first-topic-for-lecture-1}{%
\section{First topic for lecture}\label{first-topic-for-lecture-1}}

\hypertarget{second-topic-for-lecture-1}{%
\section{Second topic for lecture}\label{second-topic-for-lecture-1}}

Boing

\hypertarget{next-generation-sequencing-and-associated-conventions}{%
\chapter{Next generation sequencing and associated conventions}\label{next-generation-sequencing-and-associated-conventions}}

Preamble, preamble, preamble.

\begin{center}\rule{0.5\linewidth}{\linethickness}\end{center}

\hypertarget{required-readings-2}{%
\section*{Required Readings}\label{required-readings-2}}
\addcontentsline{toc}{section}{Required Readings}

\begin{enumerate}
\def\labelenumi{\arabic{enumi}.}
\tightlist
\item
  Write another chapter in the handbook for this
\end{enumerate}

\hypertarget{recommended-readings-2}{%
\section*{Recommended Readings}\label{recommended-readings-2}}
\addcontentsline{toc}{section}{Recommended Readings}

\begin{enumerate}
\def\labelenumi{\arabic{enumi}.}
\tightlist
\item
  This
\item
  That
\item
  More
\end{enumerate}

\hypertarget{exercises-due-by-mmddyyy-2}{%
\section*{Exercises due by MM/DD/YYY}\label{exercises-due-by-mmddyyy-2}}
\addcontentsline{toc}{section}{Exercises due by MM/DD/YYY}

\begin{enumerate}
\def\labelenumi{\arabic{enumi}.}
\tightlist
\item
  Links to notebooks that have the assignments. Will make it
  easy for students to download extra material for each one.
\end{enumerate}

\begin{center}\rule{0.5\linewidth}{\linethickness}\end{center}

\hypertarget{first-topic-for-lecture-2}{%
\section{First topic for lecture}\label{first-topic-for-lecture-2}}

\hypertarget{second-topic-for-lecture-2}{%
\section{Second topic for lecture}\label{second-topic-for-lecture-2}}

Boing

\hypertarget{unix-scripting-awk-regular-expressions}{%
\chapter{Unix scripting, awk, regular expressions}\label{unix-scripting-awk-regular-expressions}}

Preamble, preamble, preamble.

\begin{center}\rule{0.5\linewidth}{\linethickness}\end{center}

\hypertarget{required-readings-3}{%
\section*{Required Readings}\label{required-readings-3}}
\addcontentsline{toc}{section}{Required Readings}

\begin{enumerate}
\def\labelenumi{\arabic{enumi}.}
\tightlist
\item
  Handbook, chapters 5 and 6.
\end{enumerate}

\hypertarget{recommended-readings-3}{%
\section*{Recommended Readings}\label{recommended-readings-3}}
\addcontentsline{toc}{section}{Recommended Readings}

\begin{enumerate}
\def\labelenumi{\arabic{enumi}.}
\tightlist
\item
  This
\item
  That
\item
  More
\end{enumerate}

\hypertarget{exercises-due-by-mmddyyy-3}{%
\section*{Exercises due by MM/DD/YYY}\label{exercises-due-by-mmddyyy-3}}
\addcontentsline{toc}{section}{Exercises due by MM/DD/YYY}

\begin{enumerate}
\def\labelenumi{\arabic{enumi}.}
\tightlist
\item
  Links to notebooks that have the assignments. Will make it
  easy for students to download extra material for each one.
\end{enumerate}

\begin{center}\rule{0.5\linewidth}{\linethickness}\end{center}

\hypertarget{first-topic-for-lecture-3}{%
\section{First topic for lecture}\label{first-topic-for-lecture-3}}

\hypertarget{second-topic-for-lecture-3}{%
\section{Second topic for lecture}\label{second-topic-for-lecture-3}}

Boing

\hypertarget{remote-computers-and-hpc}{%
\chapter{Remote computers and HPC}\label{remote-computers-and-hpc}}

Preamble, preamble, preamble.

\begin{center}\rule{0.5\linewidth}{\linethickness}\end{center}

\hypertarget{required-readings-4}{%
\section*{Required Readings}\label{required-readings-4}}
\addcontentsline{toc}{section}{Required Readings}

\begin{enumerate}
\def\labelenumi{\arabic{enumi}.}
\tightlist
\item
  Handbook, parts of chapter 7.
\end{enumerate}

\hypertarget{recommended-readings-4}{%
\section*{Recommended Readings}\label{recommended-readings-4}}
\addcontentsline{toc}{section}{Recommended Readings}

\begin{enumerate}
\def\labelenumi{\arabic{enumi}.}
\tightlist
\item
  This
\item
  That
\item
  More
\end{enumerate}

\hypertarget{exercises-due-by-mmddyyy-4}{%
\section*{Exercises due by MM/DD/YYY}\label{exercises-due-by-mmddyyy-4}}
\addcontentsline{toc}{section}{Exercises due by MM/DD/YYY}

\begin{enumerate}
\def\labelenumi{\arabic{enumi}.}
\tightlist
\item
  Links to notebooks that have the assignments. Will make it
  easy for students to download extra material for each one.
\end{enumerate}

\begin{center}\rule{0.5\linewidth}{\linethickness}\end{center}

\hypertarget{first-topic-for-lecture-4}{%
\section{First topic for lecture}\label{first-topic-for-lecture-4}}

\hypertarget{second-topic-for-lecture-4}{%
\section{Second topic for lecture}\label{second-topic-for-lecture-4}}

Boing

\hypertarget{sequence-alignment}{%
\chapter{Sequence alignment}\label{sequence-alignment}}

Preamble, preamble, preamble.

\begin{center}\rule{0.5\linewidth}{\linethickness}\end{center}

\hypertarget{required-readings-5}{%
\section*{Required Readings}\label{required-readings-5}}
\addcontentsline{toc}{section}{Required Readings}

\begin{enumerate}
\def\labelenumi{\arabic{enumi}.}
\tightlist
\item
  Hmm\ldots{}
\end{enumerate}

\hypertarget{recommended-readings-5}{%
\section*{Recommended Readings}\label{recommended-readings-5}}
\addcontentsline{toc}{section}{Recommended Readings}

\begin{enumerate}
\def\labelenumi{\arabic{enumi}.}
\tightlist
\item
  This
\item
  That
\item
  More
\end{enumerate}

\hypertarget{exercises-due-by-mmddyyy-5}{%
\section*{Exercises due by MM/DD/YYY}\label{exercises-due-by-mmddyyy-5}}
\addcontentsline{toc}{section}{Exercises due by MM/DD/YYY}

\begin{enumerate}
\def\labelenumi{\arabic{enumi}.}
\tightlist
\item
  Links to notebooks that have the assignments. Will make it
  easy for students to download extra material for each one.
\end{enumerate}

\begin{center}\rule{0.5\linewidth}{\linethickness}\end{center}

\hypertarget{first-topic-for-lecture-5}{%
\section{First topic for lecture}\label{first-topic-for-lecture-5}}

\hypertarget{second-topic-for-lecture-5}{%
\section{Second topic for lecture}\label{second-topic-for-lecture-5}}

Boing

\bibliography{book.bib,packages.bib}


\end{document}
